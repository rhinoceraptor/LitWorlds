\documentclass[12pt, letterpaper]{report}
\usepackage[nonumberlist, toc]{glossaries}
\usepackage{listings}
\makeglossaries

\newglossaryentry{MOO}
{
	name={MOO},
	description={Multi User, Object Oriented}
}
\newglossaryentry{XP}
{
	name={eXtreme Programming},
	description={An incremental software development methodology that stresses communication, estimation, planning, testing, flexibility, and other practices. It is associated with other methodologies that implement agile development}
}
\newglossaryentry{freeSoftware}
{
	name={Free Software},
	description={Software licensed such that the user has the freedom to run, redistribute, and study the source code as they wish, and that any modifications must also be licensed under the same or compatible licenses}
}
\newglossaryentry{LARP}
{
	name={Live Action Role Play},
	description={A role playing game where players physically act as their characters, inspired by similar tabletop games}
}
\newglossaryentry{GNU}
{
    name={GNU},
    description={A free software operating system developed by the GNU project. Its software is typically bundled with the Linux kernel to form an operating system commonly called Linux. GNU is a recursive acronym, standing for 'GNU's Not Unix!'}
}
\glsaddall

\author{Owen Watson, John Lewis, Tim Cunningham}
\title{Literary Worlds Feasibility Report}
\date{March 12, 2014}
\linespread{1.5}


\begin{document}
	%-- title page --%
	\begin{titlepage}
	\Huge \maketitle \par
	\end{titlepage}
	
	%-- Abstract --%
	\chapter{Abstract}
	\par
	EnCore Literary Worlds is an interactive, networked environment for exploration of literary works. Each environment is created by an ordinary user familiar with the source work, to create a more immersive experience for other users and students. The system is designed to be very easy for new and non-technical users, as well as being complex enough for creation of manipulable objects, abilities, and interaction between users in the environment.
	
	\par
	The EnCore Literary Worlds system can be used for many genres and types of worlds, with both original or derivative works. It can be used to extrapolate source text, create second worlds, create role play or alternative reality, or even virtual tours and museums. This is explained by Allen Webb,
	
	%-- page 4 - Teaching Literature in Virtual Worlds --%
	\begin{quotation}
	While all of the literary virtual worlds described in this book [Teaching Literature in Virtual Worlds] allow group interaction, several are explicitly designed as Alternative Reality Games (ARG) activities (%
	\textit{Thoughtcrime, Midsummer Madness, The Virtual Tempest}). Other worlds (%
	\textit{The Village of Umufofia, Mice, Men, and Migrant Labor, Gatsby's American Dream})
	could be described as a virtual Live Action Role Plays (LARPs), activities typically prepared by a "gamemaster", in this case the virtual world builder (Webb 4).
	\end{quotation}
	
	\par
	This project proposes a drop in replacement for the first software release for the existing, outdated Java browser applet for interacting with the EnCore Literary Worlds system. The new system will be fully compatible with the existing script, server and backend tools and will work with common web browsers, both on desktop and mobile platforms well into the future, being built with JavaScript and other current web technologies.
	
	

	\par
    Further releases may augment the feature set offered by the browser client, update the version of the enCore software, or implement other client stories as needed.
	%-- TOC --%
	\tableofcontents

	%-- Index --%
	\chapter{Background}
	
	\section{Client}
	\par
	Allen Webb is a professor of Comparative Literature and Postcolonial Studies in the Western Michigan University Department of English. He edited the book, \textit{Teaching Literature in Virtual Worlds} (among other books, articles and presentations), and designed a literary virtual world for which he was awarded the A+ Award by Web English Teacher.
    	
	\section{Existing System}
	\par
	EnCore Literary Worlds is built on free software, using enCore 4, which itself is built from LambdaMOO, MOO standing for Multi-User Domain, Object Oriented. LambdaMOO was created in January 1991 by Pavel Curtis at Xerox PARC.\cite{Wired}
	
	\par 
	
		\par 
	The enCore MOO project was created in 1997, its aim to to create easy and convenient access to MOO systems and scripts by creating a point and click interface around the typically text based existing tools.

	\par
	The existing system uses a Java browser applet on the client end. The server end of the system runs on Mac OS X, version 10.6.8, with two 2.66 gigahertz dual core Xeon processors, four gigabytes of RAM, and several hard disks. A replacement system would likely be deployed on CentOS, Ubuntu or Debian Linux since there is no licensing fee or special hardware requirement.

	\par 
	
	
	
	%-- Stories --%
	\chapter{Stories}
	\par
	In eXtreme Programming, stories are functional descriptions in the system. The stories are rigorously defined such that there is no doubt whether or not the story is implemented correctly and completely. The stories will be written, a short description of the functionality in question. Then, the story will be estimated by the development team. Upon this feedback, the client is able to prioritize which stories get implemented first so that each release contains the most important functions.
	
	\par
	As it stands now, the claim made in "Teaching Literature in Virtual Worlds", that 
	
	\begin{quotation}
    All is needed [to access Literary Worlds] is web access and a standard browser set to 'accept popups'(Webb 9)
	\end{quotation}
	is no longer true. Java browser applets have fallen out of favor, and therefore require significant configuration on the user end to access Literary Worlds. Moving into the future, fewer and fewer users will have Java applets enabled on their desktop computers, and none of the major mobile platforms currently support a Java Runtime Environment in the browser. Therefore, a new client compatible with current and future browsers is needed.
	
	\par
	The new system will provide a secure, convenient interface to the existing system, over the telnet networking protocol. The first release of the system will strictly a feature for feature reimplementation of the Java browser applet system, using JavaScript and Node.js. This will ensure that Literary Worlds can be used by school districts, literary enthusiasts, and the general public without time consuming and potentially insecure modifications to their web browser configuration. It will also ensure that the system will be accessible through new web platforms, on smart phones and tablets of all types and operating systems.
	
	\section{Console Interface}
	The most basic implementation of the software will provide the text console interface that of original Java applet. This input will be accepted in a console interface, and will then be sent back to the telnet server running under EnCore v4, and the console response sent back to the user's terminal.
	
	\section{Graphical Interface}
	The secondary functionality of the original Java applet is the graphical interface, which displays text, images, video, audio, and other multimedia. This interface provides a more intuitive and less intimidating way for users to interact with the literary worlds. 
	
	\section{Secondary Features}
	Convenience features for the console interface could be implemented after the console and graphical systems are fully functional. Currently, the console interface does not implement color highlighting of different classes of text. It has no support for tab completion. It also does not have a feature that tries to correct typos or malformed commands. All of these would make the console interface much more convenient to use, and provide a more immersive experience.
	
	\par
	
	
	
	%-- Spikes --%
	\chapter{Spikes}
	\section{Node.js}
	

    \lstset{frame=tb,
    %language=javascript,
    aboveskip=3mm,
    belowskip=3mm,
    showstringspaces=false,
    columns=flexible,
    basicstyle={\small\ttfamily},
    numbers=none,
    numberstyle=\tiny\color{gray},
    keywordstyle=\color{blue},
    commentstyle=\color{dkgreen},
    stringstyle=\color{mauve},
    breaklines=true,
    breakatwhitespace=true
    tabsize=3
    }
	
	\par
	Node.js is a web framework platform for building event driven web applications using JavaScript. It can be used to easily write web applications, using server side JavaScript.
    	
	\par
    Node.js can be used to create trivial web applications, such as to output the text "Hello World" to a user's browser, in plaintext:
    
    \begin{lstlisting}
    var httpHelloWorld = require("http");
    httpHelloWorld.createServer(function(request, response)
    {
        response.writeHead(200, {"Content-Type" : "html/plain"});
        reponse.write("Hello World\n");
        response.end();
    }).listen(80);
    \end{lstlisting}
    
    
	\section{Debian Server}
	\par
	Debian is a distribution of GNU/Linux, which is commonly used in server and desktop deployments, it claims almost 30\% of the total web server market share.
	
	\par 
	EnCore MOO is distributed via Source Forge as a tar.gz file. A Debian virtual server was created, configured to add a user for running the MOO, root login over SSH disabled, and then EnCore MOO as well as Apache webserver installed. To do this, we need several components: a copy of enCore version 5.0, patches for enCore, and the LambdaMOO server.
    	
	%-- Legal --%
	\chapter{Legal}
	\section{Licensing}
	\par
	EnCore MOO uses the GNU General Public License, version Two free software license. It is managed by the enCore Consortium, a 501(c)3 corporation. The consortium has received several grants, one in 2006, and one in 2008.
	
	\par
	Of the many requirements mandated by the GNU GPLv2, the most important one to consider are that all derivative work that is distributed or released must also be licensed under the GPLv2 license, and typically may also be relicensed under subsequent GNU GPL licenses, such as the GPLv3 license.
	
	\section{EnCore Consortium}
	\par
    
	\par 
	From their website, the consortium's mission statement is,
	
	\begin{quotation}
	The enCore Consortium seeks to coordinate and promote the open source development and distribution of the enCore Program.
	\end{quotation}
	
	\par
    The consortium uses a board of directors administration model, and was formed in 1997. Their goal is to convert text based MOOs into user friendly, graphical interfaces that can be used without training.
	%-- Glossary --%
	%\chapter{Glossary}

	\printglossary
	%-- References --%
	%\addcontentsline{toc}{chapter}{References} % Manually add references to ToC
	%\chapter*{References}
	\begin{thebibliography}{9}

	%-- Citation - Webb - pg 64 --%
	\bibitem{Webb}
 	Webb, Allen et. al . Teaching Literature in Virtual Worlds. New York, NY: Routledge, 2012. Print.

    \bibitem{Webb Bio}
    "Allen Webb - English - Western Michigan University." . Western Michigan University. Web. 18 Mar 2014. \\
    \urlhhttp{http://www.wmich.edu/english/directory/faculty/webb.html}
    
    \bibitem{GPLv2}
    "GNU General Public License, version 2". GNU Project. Web. 18 Mar 2014. \\
    \urlhhttp{http://www.gnu.org/licenses/gpl-2.0.html}
	
	
	\bibitem{Wired}
	"Rheingold , Howard. "PARC Is Back!." \textit{Wired}. 02 1994 \\
	\urlhhttp{http://www.wired.com/wired/archive/2.02/parc\_pr.html}

	\end{thebibliography}
	
\end{document}