\documentclass[12pt, letterpaper]{report}
\author{Owen Watson, John Lewis, Tim Cunningham}
\title{Literate World Project Report}
\date{March 12, 2014}

\begin{document}
	%-- title page --%
	\begin{titlepage}
	\Huge \maketitle \par
	\end{titlepage}
	
	%-- Abstract --%
	\chapter{Abstract}
	\par
	EnCore Literary Worlds is an interactive, networked environment for exploration of literary works. Each environment is created by an ordinary user familiar with the source work, to create a more immersive experience for other users and students.
	
	\par
	This project proposes a drop in replacement for the existing, outdated Java browser applet for interacting with the EnCore Literary Worlds system. The new system will be fully compatible with the existing script, server and backend tools and will work with common web browsers, both on desktop and mobile platforms.
	
	%-- TOC --%
	\tableofcontents
	
	%-- Index --%
	\chapter{Background}
	\par
	EnCore Literary Worlds is built on open source software, using enCore 4, which itself is built from LambdaMOO. LambdaMOO stands for Multi-User Domain, Object Oriented.
	
	
	%-- Stories --%
	\chapter{Stories}
	
	\par
	As it stands now, the claim made in "Teaching Literature in Virtual Worlds", that "All is needed [to access Literary Worlds] is web access and a standard browser set to 'accept popups'" is no longer true. Java applets have fallen out of favor, and therefore require significant configuration on the user end to access Literary Worlds. Moving into the future, fewer and fewer users will have Java applets enabled on their desktop computers, and none of the major mobile platforms currently support a Java Runtime Environment in the browser. Therefore, a new client compatible with current and future browsers is needed.
	
	\par
	The new system will provide a secure, convenient interface to the existing system, over the telnet networking protocol. The first release of the system will strictly a feature for feature reimplementation of the Java applet system, using JavaScript and Node.js. This will ensure that Literary Worlds can be used by school districts, literary enthusiasts, and the general public without time consuming and potentially insecure modifications to their web browser configuration. It will also ensure that the system will be accessible through new web platforms, on smartphones and tablets of all types.
	
	
	%-- Spikes --%
	\chapter{Spikes}
	
	%-- Legal --%
	\chapter{Legal}
	\par
	EnCore MOO uses the GNU GPLv2 free software license.
	
	
	%-- Glossary --%
	\chapter{Glossary}
	\par
	MOO - Multi-User, Object Oriented
	\par
	
	
	
	%-- References - Using APA 6th edition style - citationmachine.net --%
	\chapter{References}
	\begin{bibliography}{9}

	%-- Citation - Webb - pg 64 --%
 	Webb, A. (2012). Content learning in literary virtual worlds.
  	In A. Webb, \emph{Teaching Literature in Virtual Worlds} 
  	(pp 64-81).
  	New York, NY: Routledge.

	\end{bibliography}

	
	
\end{document}